\chapter{Discussion and conclusion}
\section{Discussion}
The research question presented in the introduction states the goal of finding the best combination of whitening, filtering, and envelope detection to estimate force from sEMG. 

From the measurement results presented in figure \ref{fig:result_all_lagerrorscaling} the conclusion can be drawn that a processing chain consisting of no pre-whitening, no filtering, and infinite impulse response butterworth lowpass filter for envelope detection yields the best accuracy for force estimating. It has the lowest error rate of all techniques and introduces no lag.

This indicates that improving the signal to noise ratio of an sEMG signal does not result in a better estimation of the applied force. It should be noted that these conclusions are drawn from a single measurement at a single location and thus may yield different results in different environments. 

Another interesting observation can be made from the Lag comparison subplot in figure \ref{fig:result_all_lagerrorscaling}: only signal processing chains that use whitening appear to introduce lag. This could be caused by the phase delay that is introduced by the whitening filter. If the filter were to have a linear phase then all frequencies would be equally delayed resulting in a constant group delay. If this filter did not have linear phase but instead introduces a stronger phase delay in higher frequencies, then the total delay would become more apparent when amplifying these higher frequencies \cite{phase_delay_frequencies}. The amplification of higher frequencies does happen in the whitening filter as can be seen in the simulation in \ref{fig:whitening_simulation}, and the time domain plot shows signs of introduction of delay in the filtered signal which can stem from making higher frequencies with more phase delay more prominent. In the simulations the phase of the whitening filter was set to the phase of the input signal ('source' of filter). In the appendix are the results of constructing the whitening filter using zero-phase, linear-phase, and negative input signal phase \ref{fig:result_whitening_linearphase} \ref{fig:result_whitening_negative_sourcephase}\ref{fig:result_whitening_sourcephase}\ref{fig:result_whitening_zerophase}. Intuitively it would be expected that using negative source phase would counteract this introduced delay, but as can be seen in appendix figure \ref{fig:result_whitening_negative_sourcephase} this does not appear to be the case.

A reason why other processing steps do not seem to introduce delay when not paired with whitening could be that lag is calculated using the cross-correlation between the measured force and the estimated force. If the estimated force accurately follows the measured force, there would exist a clear peak in the cross-correlation which indicates the lag.




=> Other signals noise is larger than lag 


=> Combination of adaptive + whitening == very very bad
- instability of IIR filter

=> Beantwoorden van onderzoeksvraag

=> We make this conclusion but that only holds under these cirumstances

describe what your results mean and how they are an important contribution to the research field.

- short signal sample of MVC and noise
- no load cell measurement


What does it all mean?
◼ What hypotheses were proved or disproved?
◼ What did we learn?
◼ Why is it important (enough to report)?
\section{Conclusion}
In this report a complete signal processing chain for sEMG signals was presented. Within each step in this processing chain various processing techniques were discussed and tested to illustrate their behaviour and performance when applied to sEMG signals. 



Summarize your key findings. Include important conclusions that can be drawn and further implications for the field. Discuss benefits or shortcomings of your work and suggest future areas for research.

High-level conclusion that was not previously mentioned (The adaptive filter is better in X situation but at the cost of Y)

Summarize general trends in the data without comment, bias, or interpretation.
◼ Should add a new, higher level of analysis, and explicitly indicate the importance
of the work
◼ Do not repeat the Results section, unless special emphasis is needed
◼ Conclusions are not a summary of the work!

=> Samenvatting discussie met alleen de belangrijkste punten

\section{Recommendations}
Further research, suggestions for better results