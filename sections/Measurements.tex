\chapter{Simulation}
This section of the report describes the testing of seperate signal processing steps in a simulated environment. Each block as seen in figure \ref{fig:global_thesis_flowchart} will be tested individually, and the method and results will be discussion on a per-block basis:
\begin{itemize}
    \item Pre-whitening
    \item Filtering
    \item Envelope estimation
\end{itemize}

Unless specified otherwise, all signals in subsequent sections will be high-passed with a cutoff frequency of 1Hz to remove DC bias.

\section{Pre-whitening}
\cite{time_series_analysis_methods}


5.4.9 Zero-Padding and Prewhitening










\section{Method}

\begin{equation}
    \frac{\text{FFT}(sig \;\; after)}{\text{FFT}(sig \;\;  before)} - \frac{\text{FFT}(noise \;\;  after)}{\text{FFT}(noise \;\;  before)} = \frac{1}{1} - \frac{0}{1} = 1
\end{equation}

\textcolor{red}{Todo: Insert image of measurement setup, and example input signal with portions labeled as $noise/signal/reference_noise$}


To reliably compare different filtering techniques it is required to use a reference signal that stays consistent across simulations. Acquiring the reference signal was done by using \textcolor{red}{(INSERT HARDWARE)} attached with two electrodes to the bicep to measure a period of maximum voluntary contraction (MVC) along with a period of 0\% MVC (zero muscle contraction). The period of 0\% MVC is used as a noise floor onto which the 100\% MVC is superimposed at different amplifications. Since the frequency spectrum of an sEMG signal stays consistent across different levels of contraction \textcolor{red}{(NEED SOURCE)} it is possible to assess each filter's performance at different SNR levels.

\textcolor{red}{Insert measurement settings of Octopus}

\begin{itemize}
    \item Length of signal and noise
    \item Plot of signal and noise
\end{itemize}

\textcolor{red}{Insert flowchart of different measurement combinations}

\section{Pre-whitening}
Explain in what way pre-whitening is calculated / applied.
\section{Filters}
\subsection{Static filter}

\begin{itemize}
    \item Length of static filter
    \item Static filter type
    \item IIR/FIR? 
\end{itemize}

Explain the static filter parameters. Use fvtool to plot filter behaviour of static filters.


\subsection{Wiener filter}


\textcolor{red}{TODO: Add this book (page 538-540), got it from Martijn:
$https://websites.pmc.ucsc.edu/~fnimmo/eart290c_17/NumericalRecipesinF77.pdf$}

Explain the calculation used for the Wiener filter
Explain the steps used to calculate the coefficients for the wiener filter. Explain length of static filter. 

\subsection{Adaptive}

!! For adaptive filter use separate noise section than the sample signal uses otherwise it will have perfect SNR !!

Show the adaptive filter error over time for different test signals, response time, frequency range.

\section{Results and conclusion}

Compare how well the static and adaptive filter work, for which frequency range it can be applied


\chapter{Measurements}
This chapter of the report is concerned with comparing different techniques of estimating the EMG amplitude.

\section{Experimental setup}
The actual measuring of force data. Mention that 600g was used in an initial experiment but it yielded odd data (list possible sources: muscle contraction was very little, lot of noise in room, shield maybe not working, LOFF not working, too many channels?), so a heavier load cell was used for more correct estimation.

What did I measure, how did I measure it, where did I measure it. What did I record, what do I expect to see?

Mention hardware used (adc, load cell,) and calibration tests done.
\section{Measurement results}
Data that was NOT working, data that WAS working, notable characteristics of the data


\section{Conclusion}
Draw numeric conclusion from data, in which situations is which filter how much better?