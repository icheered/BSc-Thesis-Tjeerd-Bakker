\chapter{Measurements}
This chapter aims to validate the accuracy of force estimation from sEMG after different processing techniques by comparing it to measured data. The goal is to measure sEMG from biceps and triceps, and record sEMG reference noise, and measure the estimated force using a load cell during an exercise of isometric contraction. The sEMG data is then processed using the different techniques that are discussed in the simulation chapter, and the final estimated force will be compared to the measured force. 

\section{Experimental setup}
The measurement setup consists of the following components:
\begin{itemize}
    \item Siemens Single Point Load Cell, 20kg Range, Compression Measure
    \item Keysight E3631A DC power supply at 2V to power the load cell
    \item TMSi Refa8-16e 16 channel amplifier
    \item Kendall H124SG Foam-Hydrogel ECG Electrodes 
\end{itemize}




Load cell with handle
Power supply
Multi-channel amplifier




The actual measuring of force data. Mention that 600g was used in an initial experiment but it yielded odd data (list possible sources: muscle contraction was very little, lot of noise in room, shield maybe not working, LOFF not working, too many channels?), so a heavier load cell was used for more correct estimation.

What did I measure, how did I measure it, where did I measure it. What did I record, what do I expect to see?

Mention hardware used (adc, load cell,) and calibration tests done.
\section{Measurement results}
Data that was NOT working, data that WAS working, notable characteristics of the data


\section{Conclusion}
Draw numeric conclusion from data, in which situations is which filter how much better?