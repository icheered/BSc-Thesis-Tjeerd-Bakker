%\chapter*{Useful definitions}
\section{List of symbols}
This table contains an overview of the symbols used in this work, their associated meanings, and their units.

\begin{table}[H]
    \centering
    \begin{tabular}{p{0.15\linewidth} | p{0.4\linewidth} | p{0.4\linewidth}}
    Symbol & Definition & Unit \\ \hline
    $f$ & Frequency & Hertz (Hz) \\ 
    $f_\text{cut}$ & Cut-off frequency & Hertz (Hz)
    \end{tabular}
    \caption{Symbol definitions}
    \label{tab:symbol_definitions}
\end{table}

\section{List of medical terms}
A list of medical terms is given because the reader is expected to be an electrical engineer and not a medical student.
\begin{table}[H]
    \centering
    \begin{tabular}{p{0.25\linewidth} | p{0.7\linewidth}}
    Term & Definition \\ \hline
    Skeletal Muscles & Muscles that are used to control voluntary body movement \\
    Flexor & A muscle that when contracted causes the angle between bones connecting to a joint to decrease (e.g. a Bicep) \\
    Extensor & A muscle that when contracted causes the angle between bones connected to a joint to increase (e.g. a Tricep) \\
    Antagonistic Muscles & A set of a flexor and extensor that have the ability to freely move a limb around a joint \\ 
    Isometric Contraction & A muscle contraction that does not result in a change of joint angle (e.g. the joint is blocked or antagonistic muscles contract simultaneously)\\
    Isotonic Contraction & A contraction of antagonistic muscles that causes the angle of the joint to change \\
    \end{tabular}
    \caption{Medical terms}
    \label{tab:medical_definitions}
\end{table}