\chapter{Theory}
I will attempt to explain all theory required to understand this project using a top-down approach. By first explaining the big picture in which this research project is located we hope to avoid the reader getting lost in the details and misunderstanding the purpose of this report. 

\section{Force estimation}
When moving a limb the most intuitive way of describing it is a change in position, moving your hand from A to B. However, a more objective way of describing this movement is in terms of forces applied on a mass:
\begin{itemize}
    \item Muscles contract
    \item This causes a force to be applied on a mass, or a torque around a pivot point
    \item This force results in an acceleration in a certain direction
    \item This acceleration is maintained for a certain period of time
    \item The entire process is repeated for deceleration using visual feedback for fine tuning of forces
    \item Your limb has arrived at a new location.
\end{itemize}

Understanding this reasoning of moving a limb in terms of forces being applied by contracting muscles is vital because it forms the basis for recognizing a user's intent. EMG can be used to measure the degree of contraction of a muscle, and by measuring the degree of contraction of two antagonistic muscles it is possible to calculate how much force is applied in a certain direction which signals a desire for this limb to move. Even if the limb is not present and replaced by a prosthesis this idea of forces moving a mass will still apply, and thus EMG can be used as a human-machine interface.

So to summarize the basic concept of force estimation:
\begin{itemize}
    \item Movement of a limb is the result of a force acting on that limb
    \item This force is the result of certain muscles contracting stronger than other muscles
    \item The contraction of these muscles can be measured using EMG
    \item EMG can be used to estimate limb movement
\end{itemize}

\section{sEMG signal properties}
\textcolor{red}{< IMAGE OF SEMG SIGNAL DURING CONTRACTION WITH ENVELOPE >}

\textcolor{red}{< SUBPLOT OF FREQUENCY SPECTRUM OF SEMG SIGNAL AND FREQUENCY SPECTRUM OF NOISE>}

\textcolor{red}{< IMAGE OF ANATOMY OF MUSCLE >}


\textcolor{red}{(TODO: ADD SOURCE!!!, ADD FIGURES, REFER TO FIGURES IN TEXT)}
A large skeletal muscle such as the bicep consists of hundreds of thousands of small muscle fibers. These muscle fibers are divided into groups called motor units, and each motor unit is connected to a motor neuron which is a special type of very long brain cell that runs through the spinal cord. A contraction of a skeletal muscle is the result of many muscle fibers contracting individually and repeatedly. The contraction of these muscle fibers is the result of an action potential caused by the motor neuron, and this action potential is a measurable (but very small) voltage. When measuring the surface EMG of a contracting skeletal muscle the result is the aggregate of the small voltages from all contracting muscle fibers. This manifests itself into a signal resembling white noise where the amplitude of the noise correlates to the number of contracting muscle fibers and thus to the amount of contraction the skeletal muscle experiences.

So to summarize: to determine the degree of contraction of a skeletal muscle we simply need to determine the amplitude of the noise at the surface of the muscle.

From this point onwards I will refer to the 'noise' generated by muscle contraction as 'the signal'. This is done because noise is usually unwanted, but the signal caused by muscle contraction is the opposite of unwanted: It precisely what we're trying to measure! 

Sadly, when measuring sEMG signals it is impossible to measure solely the signal generated by muscle contraction. The signal may be polluted by other signals coming from the surrounding environment (such a 50 Hz power lines nearby) or from the amplifier used to amplify the measured signal. So in reality we are measuring a combination of our desired signal from muscle contraction, and the undesired noise from the environment and amplifier.

The ratio between how much of the measured signal is actual desired signal and how much is undesired noise is called the Signal to Noise Ratio (SNR) and is defined as the signal power divided by the noise power \textcolor{red}{(SOURCE) (EQUATION)}. As the amount of desired signal increases and the amount of noise decreases, the accuracy with which the force can be estimated from the contraction also increases. In other words, a larger SNR results in a more accurate estimation. We can increase the SNR by removing noise, and for this exact purpose 'filters' were invented.

\section{Filters}
% Introduce basic concept of filters
At a fundamental level filters are simply a tool to remove something unwanted (noise) that is mixed with something wanted (signal). In signal processing filtering is achieved by decomposing a measured signal into repeating patterns and subsequently deciding which patterns should be included and which patterns should be removed. 

\textcolor{red}{< Image of signal and FFT of signal >}

% Explain difference between analog digital and continuous and discrete
The physical world is inherently analog. Any property that changes from one value to another (such as a change in as wind speed, water level, voltage or age) will at some point have been equal to any value between the start and end value; Between 1 and 2 Volt there are infinite values with infinite decimals \textcolor{red}{(SOURCE)}. Computers sadly don't have infinite memory to store all these values and therefore to measure a continuous signal it needs to be sampled at equidistant points in time. This turns an analog continuous signal into a digital discrete signal. Even though filtering is possible in the analog domain, it is much less complicated and more flexible to filter in the digital domain \textcolor{red}{(NEED CITATION)}.

\textcolor{red}{< Image of analog vs digital as illustration of above text >}

In the digital domain the basic concept of a filter is as simple as it is magical. A filter (usually) consists simply of a set of values called the filter coefficients. The input (measurements) is multiplied with the filter coefficients to create the output. That is, the latest measurement is multiplied with the first filter coefficient, the previous measurement is multiplied with the second filter coefficient, and so on. By carefully choosing the number and value of filter coefficients it is possible to attenuate specific frequencies while not influencing other frequencies. (REFER TO FFT PICTURE)

\textcolor{red}{< Image of simple digital filter as illustration for above text >}

\subsection{Static filters, Wiener filter, Adaptive filters}
If a filter is static (e.g. high-pass, low-pass, band-pass, or band-stop) it simply means that the amount of filter coefficients and the values of the filter coefficients are predetermined. These filters are very popular due to their simplicity in terms of finding the value of the filter coefficients. If a signal is expected to be between 90 and 100 Hz, a bandpass filter can be created even before the signal is measured that removes all signal components \textit{except} those between 90 and 100 Hz just by multiplying some values together. 

\textcolor{red}{< IMAGE WITH FREQUENCY RESPONSE OF STATIC FILTER > }

A Wiener filter is a static filter that removes the frequency contents of noise from the frequency contents of a signal. In other words, the Wiener filter attenuates the frequencies that are present in noise so that when the filter is applied to a signal that is polluted by the same noise it effectively removes the noise. This is done by measuring two things: The signal that is corrupted by noise, and the noise separately. For example when measuring an sEMG signal an additional channel is used to measure the signal of a non-contracting muscle to gather only the present noise. \textcolor{red}{(SOURCE)}
The Wiener filter requires both the signal and the noise to be stationary, i.e. the spectral density does not change over time, and results in a linear time-invariant filter. \textcolor{red}{((Wikipedia, stationary process), (https://askanydifference.com/difference-between-stationary-and-non-stationary-signals/))}
\textcolor{brown}{\begin{itemize}
    \item Wiener-Hopf Equation
    \item Mean Square Error definition
    \item Leas Square Error algorithm
\end{itemize}}

\textcolor{red}{< IMAGE WITH FREQUENCY RESPONSE OF WIENER FILTER > }

Adaptive filters expand on the idea of using knowledge of the spectral domain to create a filter by only using a subset of the input data (last $n$ samples). By re-calculating the filter coefficients with new data it is possible to update the filter to match a changing spectral density. 

Adaptive filters are applied in the same way as static filters by just multiplying the filter with measurements. However, the  method of finding the filter coefficients is different. Instead of deciding beforehand what frequencies you want to have removed, you measure the noise and determine its frequency spectrum. Then you tune the filter coefficients to remove these frequencies from the measurements (which also includes noise) leaving only the desired signal. This filter is used in environments where the noise frequency spectrum is not known beforehand, or when the frequency behaviour of the noise may change over time. Common applications of adaptive filters include speech recognition and noise-cancelling headphones!
\textcolor{red}{($https://www.researchgate.net/publication/330902230_Performance_of_Wiener_Filter_and_Adaptive_Filter_for_Noise_Cancellation_in_Real-Time_Environment$)}

\textcolor{red}{(TODO: MATHEMATICAL DEFINITION OF FINDING COEFFICIENTS IN ADAPTIVE FILTERS) }

\textcolor{blue}{
=> LMS algorhitm
=> """"Instead of computing as suggested by
Wiener-Hopf equation, in LMS the coefficients are adjusted
from sample to sample in such a way as to minimize the MSE.... The LMS algorithm is based on steepest descent algorithm. ... """" => Wiener-hopf algorithm calculates $W_{opt}$ (Performance of Wiener Filter and Adaptive Filter for Noise Cancellation in Real-Time Environment, paper).}

\textcolor{red}{< IMAGE OF ADAPTIVE FILTER DIAGRAM >}

\subsection{FIR vs IIR}
Another subdivision within filter design is concerned with the type of possible responses to a specific input (impulse) and whether or not this can go to infinity.

The previously discussed filters were all described as Finite Impulse Response (FIR) filters. This means that the output is the result of multiplying the filter coefficients with the input. This type of filter is always stable and the output can never go to infinity as long as the input does not go to infinity.

An Infinite Impulse Response filter calculates the output using two sets of filter coefficients. The first set of filter coefficients is used to multiply with the input just like a FIR filter, but the second set of filter coefficients is used to multiply with previous \textit{outputs}. This means that there is now a feedback loop in the system, and a system with feedback can become unstable. Unstable in this case means that there is a possibility of positive feedback loop where increasing output values result in future output values also increasing, eventually going to infinity. Even though this feedback and possible instability may sound like a downside, it also results in shorter filter length and thus fewer computations required per filter operation. This could especially provide beneficial in low memory and low compute power environments like in prostheses \cite{fir_vs_iir}.

Both static and adaptive filters can be implemented as both FIR or IIR filters. An adaptive IIR filter offers the potential to meet desired performance levels with much less computational complexity. However, the possibility for the system to become unstable combined with the fact that filter coefficients are adjusted automatically leads to a high-risk high-reward scenario due to a loss of control and hard to predict behaviour.  \textcolor{red}{(Boek over adaptive IIR filters: http://dsp-book.narod.ru/DSPMW/23.PDF)}

\textcolor{red}{(TODO: MATHEMATICAL DEFINITIONS OF FIR/IIR/ADAPTIVE)}

\section{Pre-whitening}

\textcolor{red}{Todo: This source was a revelation:

https://www.youtube.com/watch?v=R4RpNG_Botk&ab_channel=DavidDorran}


The measured sEMG signal is not white. That is, not all frequencies are equally present in the signal. Certain frequencies present in the sEMG signal will have much more spectral power than other frequencies. However, these powerful frequencies may be not be proportionally correlated with the degree of muscle contraction. Other frequencies with less power in the spectral domain may carry more information as to the degree of contraction.

\begin{quote}
    "Prewhitening is a filtering or smoothing technique used to improve the statistical reliability of spectral estimates by reducing the leakage from the most intense spectral components and low-frequency components of the time series that are poorly resolved. To reduce the biasing of these components, the data are smoothed by a window whose spectrum is inversely proportional to the unknown spectrum being considered. 
    
    Prewhitening reduces leakage and increases the effectiveness of frequency averaging of the spectral estimate (reduces the random error). The reduced leakage gives rise to a greater dynamic range of the analysis and allows us to examine weak spectral components".
    (\cite{time_series_analysis_methods_whitening}, Chapter: 5.4.9. Zero-Padding and Prewhitening)
\end{quote}

\textcolor{red}{(TODO: INSERT MATHEMATICAL DEFINITION)}

\section{Amplitude estimation}
As mentioned earlier the amplitude of a measured sEMG signal scales linearly with how much a skeletal muscle is contracted. \textcolor{red}{(REFER TO FIGURE)} If we mirror the negative part of the sEMG signal onto the positive part (i.e. taking the absolute value), a line can be drawn from peak to peak to create a smooth curve that represents the contraction of a muscle. This is called an envelope.

Computationally envelope detection can be achieved in a number of different methods where "speed", or how much the detected envelope lags behind the true signal envelope, is traded against noise. \textcolor{red}{(https://www.dsprelated.com/showarticle/938.php) \url{https://www1.udel.edu/biology/rosewc/kaap686/notes/EMG%20analysis.pdf
}}.

Most conventional envelope detection methods calculate the value of a sample based on samples before \textit{and after} the measured point. This is possible in non-real-time signal processing, but during real-time signal processing comes at the cost of a delayed output signal.

\textcolor{red}{< INSERT IMAGE OF DIFFERENT DETECTION TECHNIQUES >.}

\textcolor{red}{(NICE PICTURE ON WIKIPEDIA:  $https://en.wikipedia.org/wiki/Envelope_detector$) }

Maybe compare behaviour in fast and slow muscle contractions?

\textcolor{blue}{""""The Influence of the sEMG Amplitude Estimation Technique on
the EMG–Force Relationship (paper)"""" <= This paper explains difference between static and adaptive amplitude estimation techniques
""""As a general consideration, the slowly varying nature of the force-tracking experiments
that have been analyzed here puts the slower filters in a clear advantage with respect to the shorter time windows; moreover, the nature of the quality parameter (i.e., the correlation) is intrinsically higher for time signals that come from very long time windows that are consequently very smooth. Even with this advantage, and even by estimating an envelope that is less smooth, MWADA is comparable to those optimal solutions; as a consequence of this result, it is reasonable to suppose that the behavior of the adaptive algorithm is more consistent across different scenarios, in which the amplitude of the sEMG signal is varying with faster dynamics""""}

\subsection{Static}
\textcolor{red}{(TODO: INCLUDE STATIC ENVELOPE DETECTION METHOD INFORMATION)}

\textcolor{red}{TODO: insert this source, it includes a lot of info about envelope detection https://www1.udel.edu/biology/rosewc/kaap686/notes/EMG%20analysis.pdf
}

\textcolor{brown}{\begin{itemize}
    \item Moving average
    \item Low-pass filter
    \item RMS
\end{itemize}}

\subsection{Adaptive}
\textcolor{brown}{\begin{itemize}
    \item higher bandwidth detection for 'faster' response
\end{itemize}}

\textcolor{red}{(TODO: INCLUDE ADAPTIVE ESTIMATION INFORMATION WITH SOURCES)}


\section{Standard sEMG signal processing}
A conventional static sEMG signal processing chain is described in \cite{muscle_force_estimation}. The relevant steps are as follows:
\begin{itemize}
    \item Highpass filter to remove DC
    \item Bandpass filter 20-300 hz
    \item Notch filter at 50hz
    \item Half-wave rectification
    \item Lowpass filter for envelope detection
\end{itemize}

These steps constitute one of the methods that will be tested in this report.