\chapter*{Abstract}

Surface Electromyography (sEMG) is the technique of measuring the electrical activity that forms on the skin surface in in response to a muscle contraction. sEMG signals can be used to determine the status of muscles and the movement intention and is therefore commonly used as a control interface for robotic prosthesis or in the medical field to monitor muscle activity. To convert the measured surface potential into a usable signal, the data needs to be processed to filter out noise and determine the envelope. Even though a lot of research has been done into various processing techniques, a general overview comparing the differences in performance of these techniques is lacking. This report aims to give insight into the degree of effectiveness of pre-whitening, various filtering methods, and multiple envelope estimation techniques. The goal is to find a combination of methods that can accurately estimate exerted force from the sEMG signals in real-time. The filtering methods that are compared are a static filter (bandpass and notch filters), a Wiener filter, and an adaptive LMS filter. The envelope detection methods that are compared are a moving average filter, an infinite impulse response Butterworth low-pass filter, and a root-mean-square filter. Each method has been applied in a simulated environment to determine the optimal parameters, and every combination of pre-whitening, filtering, and envelope detection has also been applied on a measured sEMG sample. The resulting force estimates are subsequently compared to force measurements obtained using a calibrated load cell. The measurement results indicate that using adaptive filtering using the LMS algorithm combined with RMS envelope detection result in predicting the exerted force approximately half a second before it is measured. Whitening does not seem to improve the quality of force estimation and introduces consistent lag compared to non-whitened processing methods.

Keywords: sEMG, force estimation, signal processing
