%% Use this line for the final version of your report
\documentclass[final,american]{include/RaM/RaM-MScReport}

%% Use this line for the draft versions of your report, it enables rro/notes/line numbers/date in footer
%\documentclass[lineno,UKenglish]{include/RaM/RaM-MScReport}

\settitle{Comparing processing techniques for real-time force estimation from sEMG}
\setauthor{Tjeerd Bakker}
%-------------------------------------------------------------------------------------------------
% This settings.tex contains settings required for *all* documents (reports, presentations, etc)
% Project or Report specific settings should go to their own settings files (eg CE/settings.tex)
% This file is included after the class definition and before project and report specific settings 
%-------------------------------------------------------------------------------------------------

%--------Useful packages (required by the example files, turn off if you do not use them)-------
\usepackage{babel}					% Add language specific support
%\usepackage{makeidx}				% Index support
%\usepackage[totoc,justific=RaggedRight]{idxlayout}	% Make last page of index balanced and add index to toc
\usepackage{caption}				% Provides means to style captions

\DeclareCaptionType{equ}[][]
%\captionsetup[equ]{labelformat=empty}

%\usepackage{subcaption}				% Provides support for (sub)figures and (sub)tables
%\usepackage{float}					% Improved interface for floating objects (eg figures, tables, ...)
\usepackage{enumitem}				% Add styling support to (enumerate) environments
\usepackage{listings}				% Allows (external) source files to be shown in a syntax highlighted way
\usepackage{amsmath}				% Provides miscellaneous enhancements for documents containing formulas
\usepackage{datetime}				% Provides commands for displaying the current time
%\usepackage{etoolbox}				% Provides \AtBeginEnvironment command
\usepackage{eurosym}				% Defines \euro command to display euro symbols
\usepackage{siunitx}
\usepackage{hyperref}
\usepackage{float}

%\usepackage{appendix}				% Makes it possible to modify appendix numbering
%\usepackage{longtable}				% Allows tables to span multiple pages
%\usepackage{units}					% Shows units (eg m/s) in a nice way
%\usepackage{ctable}				% Provides \ctable command for the typesetting of table and figure floats
%\usepackage{ccaption}				% Support continuation captions (eg multi-page tables)
%\usepackage{verbatim}				% Adds verbatim environment, in which texts are exactly copied to the output
%\usepackage{pdfpages}				% Include PDF pages/documents in the current document
\usepackage{include/files/requirements}

\iffinalversion
	\usepackage[final]{include/files/notes}% Add note commands, [final] removes all notes from the document
	\usepackage[final]{include/files/rro}  % Add Rich Report Outline support, [final] removes all RRO output from document
\else
	\usepackage{include/files/notes}       % Add note commands
	\usepackage{include/files/rro}         % Add Rich Report Outline support
\fi

% Add wrongly (or unknown) hyphened words here (space separated and - at possible hyphenation positions):
%\hyphenation{}

%% Spacing possibilities for captions are available as well
% See captions.pdf for all options!
\captionsetup{font=small,labelfont=bf}

%% Center all figures by default
%% http://tex.stackexchange.com/questions/2651/should-i-use-center-or-centering-for-figures-and-tables
\makeatletter
\g@addto@macro\@floatboxreset\centering
\makeatother

%% Make use small font size in verbatim environment
% Note: AtBeginEnvironment is provided by etoolbox package
%\AtBeginEnvironment{verbatim}{\small}

%% Include verbatim in the subfigure env
% From: subfig.pdf, section 4.4
% <Uncomment if verbatim is required in subfloat>:
%\makeatletter
%\newbox\sf@box
%\let\orig@subfloat\subfloat
%\renewenvironment{subfloat}[2][]%
%{ \def\sf@one{#1}%
%  \def\sf@two{#2}%
%  \setbox\sf@box\hbox
%  \bgroup}%
%{ \egroup
%  \ifx\@empty\sf@two\@empty\relax
%    \def\sf@two{\@empty}
%  \fi
%  \ifx\@empty\sf@one\@empty\relax
%    \orig@subfloat[\sf@two]{\box\sf@box}%
%  \else
%    \orig@subfloat[\sf@one][\sf@two]{\box\sf@box}%
%  \fi}
%\makeatother
%% Uncomment till here
  
%% Automatically provide H option for floats
% Requires float package
% \floatplacement{figure}{H} 
% \floatplacement{table}{H} 

%% abbreviation making
\newcommand{\abbr}[1]{(\textit{#1})}

%%lstlisting settings
\lstset{	%aboveskip=20pt,%
		numbers=none, %no line numbers
%		numbers=left, %show line numbers
		numberstyle=\tiny,%
		frame=single,%
		frameround={t}{t}{t}{t},%
		numbersep=5pt,%
%		language=C,% (default) code language in the document
		captionpos=b,%
		xleftmargin=2em,
		framexleftmargin=1.5em,
		xrightmargin=2em,
		framexrightmargin=1.5em,
		morecomment=[s][\itshape]{<}{>}, %also define <> as comment
		morecomment=[s][\itshape]{[}{]} %also define [] as comment
}

%lstinline with empty language definition
\lstdefinelanguage{empty}{}
\newcommand{\mylstinline}[1]{{\lstinline[language=empty]{#1}}}

% Default value of top separator (empty space) of lists
\setlist{topsep=4pt}

\bibliographystyle{IEEEtran}

%% Don't show warnings like: ``PDF inclusion: found PDF version <1.x>, but at most version <1.4> allowed
% Uncomment if you experience these kind of warnings 
%\ifpdf
%	\pdfminorversion=6 
%\fi

%%%%%%%%%%%%%%%%%%%%%%%%%%%%%%%%%%%%%%%%
% Macros for commenting and correcting.
%%%%%%%%%%%%%%%%%%%%%%%%%%%%%%%%%%%%%%%%
\usepackage{xcolor}															%Needed to have some colored text
\usepackage{soul}															% Strike out text using \textst{...}
\newcommand{\crgk}[1]{\noindent\textcolor{blue}{{#1}}} 							%Corrections	
\newcommand{\cmgk}[1]{\noindent\textcolor{blue}{\emph{GK:} #1}}  					%Comments
\newcommand{\rmgk}[1]{\noindent\textcolor{red}{{\textst{#1}}}} 						%Remove
\newcommand{\rpgk}[2]{\noindent\textcolor{red}{{\textst{#1 }}\textcolor{olive}{{#2}}}}  	%Replace
\newcommand{\gk}[1]{\noindent\textcolor{red}{#1~Please: reread, reconsider, rephrase!}}
\newcommand{\csgk}{\cmgk{Clumsy sentence, please rephrase.}}     					%Improve sentence
\newcommand{\itgk}{\cmgk{Illegible text in graph. Please improve.}}  					%Improve text in figure
\newcommand{\wngk}[1]{
\vspace{4mm}
\fbox{\begin{minipage}{0.90\columnwidth}\noindent\textcolor{red}{#1}\end{minipage}}
\vspace{4mm}
}



\begin{document}
% Numbered roman style

\frontmatter

% this is just a temporary front page. You will can get the final front page from Jolanda when you are close to finishing your report.

% Use \maketitle or the available PDF when it is released (for student reports)
%\maketitle

% Enter the name of the official RaM title page PDF between the brackets
% ! This method disables the option of using EPS files in your report.
% ! If EPS images are required, use LaTeX source instead of the PDF file
\includepdf{032bakker2022.pdf}


\cleardoublepage

\cmgk{General remarks:
\begin{itemize}
    \item Please adjust title page. Massimo is professor.
    \item Use non-italic font for subscripts indicating a name since it improves readability. E.g. $\phi_\text{C}$ instead of $\phi_C$, unless the subscript represents a variable, e.g. $\phi_i=i\cdot\Delta\phi$ would e.g. be the potential of the $i^{th}$ element. Also use the \LaTeX~names for functions. So `$\tan$' instead of `$tan$' since `$\tan$' is actually a name (of a function), not a variable.
    \item Considering the use of "a" and "an": "an" is used for all words and abbreviations that start with a \emph{vowel}. This does include the consonants that, when spoken out loud, begin with a vowel sound. E.g. "an SEM picture" since SEM is pronounced "es-ee-em". But one would use "a DAC", pronounced "de-a-c".
    \item Use \texttt{siunitx} to obtain very consistent type setting of numbers and units. A single value would e.g. look like \SI{2.8}{\ohm\cm} (\texttt{\tb SI\{2.8\}\{\tb ohm\tb cm\}}. A range \SIrange{2.8}{9}{\ohm\cm}. Note that it takes care to consistently typeset number, unit and spacing. For one thing, units should not be typeset italic!
    \item Try to keep figures and corresponding text close to each other; scrolling through all the pages is not convenient for your readers. If needed you can use \texttt{FloatBarrier} to prevent figures and tables to be moved past a certain point. 
    \item Use \emph{amount, lot of} for uncountable quantities, \emph{many, number of} for countable quantities.
\end{itemize}
}

\chapter*{Abstract}
\textcolor{brown}{
This bachelor thesis investigates to which extend adaptive filters can improve digital signal processing of surface Electromyography signals compared to standard static filters. < Insert more stuff once I finish the report > } \\


% Add the table of contents pages (TOC)
\tableofcontents

% The report starts here

\mainmatter

%\chapter*{Useful definitions}
\section{List of symbols}
This table contains an overview of the symbols used in this work, their associated meanings, and their units.

\begin{table}[H]
    \centering
    \begin{tabular}{p{0.15\linewidth} | p{0.4\linewidth} | p{0.4\linewidth}}
    Symbol & Definition & Unit \\ \hline
    $f$ & Frequency & Hertz (Hz) \\ 
    $f_\text{cut}$ & Cut-off frequency & Hertz (Hz)
    \end{tabular}
    \caption{Symbol definitions}
    \label{tab:symbol_definitions}
\end{table}

\section{List of medical terms}
A list of medical terms is given because the reader is expected to be an electrical engineer and not a medical student.
\begin{table}[H]
    \centering
    \begin{tabular}{p{0.25\linewidth} | p{0.7\linewidth}}
    Term & Definition \\ \hline
    Skeletal Muscles & Muscles that are used to control voluntary body movement \\
    Flexor & A muscle that when contracted causes the angle between bones connecting to a joint to decrease (e.g. a Bicep) \\
    Extensor & A muscle that when contracted causes the angle between bones connected to a joint to increase (e.g. a Tricep) \\
    Antagonistic Muscles & A set of a flexor and extensor that have the ability to freely move a limb around a joint \\ 
    Isometric Contraction & A muscle contraction that does not result in a change of joint angle (e.g. the joint is blocked or antagonistic muscles contract simultaneously)\\
    Isotonic Contraction & A contraction of antagonistic muscles that causes the angle of the joint to change \\
    \end{tabular}
    \caption{Medical terms}
    \label{tab:medical_definitions}
\end{table}

\chapter{Introduction}
\section{Context}
Electromyography (EMG) is the process of measuring the electrical activity that forms in a muscle in response to a nerve stimulating the muscle fibers \cite{biomechanics_research_methods} \cite{wikipedia_emg}. EMG is an popular method of measuring a person's intent to contract a muscle as it measures the muscle activation rather than the muscle contraction \cite{control_interfaces_intention_detection}. This means that it can still be used in scenarios where muscles can not respond accurately to nerve stimulation due to for example muscular dystrophy \cite{emg_arm_function_boys_pilot}. As a result, EMG is a good way of creating a control interface for exo-aids in various scenarios. Additionally, the amplitude of the EMG signal has a roughly linear relation with the force produced by the muscles and is therefore suitable for human machine interfaces \cite{adaptive_filter_dry_electrode}.

The electrical activity of a muscle can either be measured by probing the inside of a muscle (called intramuscular EMG or iEMG), or by measuring the electrical potential on the surface of the skin (called surface EMG or sEMG). iEMG has a high selectivity for individual motor neuron units which is desirable for a precise control interface registering multiple degree of freedoms \cite{semg_vs_iemg}, but has as a downside that it is an invasive and difficult procedure \cite{intramuscular_emg_signals}. sEMG is a non-invasive method of measuring requiring only sticking electrodes on the skin but this method can only measure the combined electrical activity of many muscle fibers resulting in a noisy and imprecise signal \cite{wiki:Electromyography}. 

During the last 2 decades research has attempted to gather more accurate sEMG readings by increasing the number of electrodes on a muscle with a technique called high-density sEMG \cite{high_density_semg}. 
This technique has allowed the measuring of spatial muscle activation in addition to time domain muscle activation. By measuring the muscle activation of a single muscle at multiple points it is theoretically possible to determine the behaviour of individual motor units \cite{high_density_semg_clinical_applications}.

However, this increased accuracy comes with a catch: Each electrode outputs a single data stream that needs to be processed. Adding more electrodes means having to process more data requiring faster, more efficient, and more accurate data processing techniques to create a system that can provide support in real time.

This project aims to give an overview of the effectiveness of different EMG processing techniques. The effect of whitening the input signal, different filtering techniques, and different envelope detection techniques are discussed, with the overarching goal of performing force estimation from sEMG signals.

\section{Related work}
There are a number of works on applying different filtering techniques in the medical field. Some interesting papers that closely relate to this assignment are discussed.

An example that shows the effectiveness of adaptive filters for real-time signal processing is \cite{adaptive_filter_emg_noise_cancellation_ecg} which covers the removal of an EMG signal from an electrocardiogram (ECG, electrical activity around the heart) signal. This is notable because the signal spectra of EMG and ECG overlap to a large extend and are therefore notoriously hard to remove using static filters. Furthermore it is mentioned in \cite{influence_semg_amplitude_estimation_technique_on_emg_force_relationship} that adaptive filters might be the most suitable type of filters for estimating force from sEMG signals.

The performance of adaptive filters and Wiener filters for noise cancellation in real-time environments in general has been tested in \cite{wiener_vs_adaptive_realtime_noisecancellation}. This report provides a solid groundwork for an intuitive understanding of wiener filters and adaptive filters which are used in this project.

A bold but promising implementation of Wiener filter in sEMG is presented in \cite{wiener_filter_a_priori_semg}. This paper presents the problem of voluntary EMG signals being contaminated by spontaneous unwanted motor activity from paretic muscles in for example stroke or spinal cord injury patients. Since EMG signals from muscles (regardless of which muscles) present similar behaviour in the frequency spectrum \textcolor{red}{(NEED SOURCE ON THIS)} it is nearly impossible to determine whether muscle activity originated from the target muscle or adjacent muscles. The research uses an 'a priori' SNR (deduced from theory, not measurements) to filter the EMG signal from the involuntary muscle contractions. 


A common technique of improving signal to noise ratio (SNR) through pre-processing is 'whitening'. Whitening decorrelates the sEMG signal to yield improved signal accuracy \cite{emg_whitening}. 
Whitening can primarily be used for high-amplitude EMG signals and has trouble retaining effectiveness on low amplitude signals. This problem has been attempted to solve by creating adaptive whitening filter which shows promising results when applied to low EMG amplitude signals \cite{adaptive_whitening}. 

An important paper in the field of EMG signals is \cite{optimal_myoprocessor}. This research assignment aims to provide a fully mathematical solution for an optimal myoelectric signal processor, and to do so a phenomenological mathematical model of myoelectric activity is presented. In other words: This paper uses biology, physics and statistics to create a formula that predicts the EMG signal from muscle activity.

There are several papers related to signal processing of sEMG signals.
One outstanding example is \cite{adaptive_filter_dry_electrode} from 2019, where it is shown that adaptive filters are an effective solution to EMG signal processing. The paper presents the instrumentation scheme of a dry-electrode sEMG measurement setup and explains a method of creating real-time adaptive finite impulse response (FIR) filters. In essence it is very similar to what my research project attempts to achieve, but the paper you are reading right now is aimed at comparing the performance between filters rather than the creation of one.

Lastly I would like to mention that static and adaptive filters are only a subset of the available tools for sEMG signal processing. An emerging topic in this field is the use of machine learning methods for signal analysis. Applying machine learning could allow for very quick, efficient, and effective signal processing, with the downside of unexpected behaviour in certain situations. Where static filters and adaptive filters exhibit deterministic behaviour (which is difficult to confirm in the case of real-time adaptive filters), machine learning methods can be somewhat 'hit or miss'. \cite{ml_semg_processing_1}.
\cite{ml_semg_processing_2}
"In many cases, ML model predictions have been found to be objectionable and violating their original expectations after deployment. A key reason is that ML models are often complex black-boxes and thus have varying, unknown failure modes that are revealed only after deployment" \cite{microsoft_machine_learning_reliable}.

\section{Research goal}
Even though a significant amount of research has been done into the digital processing of sEMG signals, an overview comparing different techniques is lacking. This report aims to give an overview of pre-whitening, different filtering techniques, and different envelope detection techniques. The overaching goal is comparing each method with the goal of \textit{real-time} force estimation. 
In figure \ref{fig:global_thesis_flowchart} a high-level overview of the signal processing chain is presented. 

\begin{figure}[h!t]
	\begin{center}
		\includegraphics[height=60mm]{images/global_thesis_flowchart.png}
	\end{center}
	\caption{High-level overview of the signal processing chain}
	\label{fig:global_thesis_flowchart}
\end{figure}


\section{Approach}
The approach for this project consists of two main steps: Evaluate filter performance on sEMG signals, and evaluate the influence of whitening and different envelope extraction techniques on the lag and error of an estimated force signal.

The filters are assessed by signal to noise ratio (SNR) and signal integrity. Different envelope estimation techniques are evaluated by their lag and error compared to a simulated modulation. The combination of whitening, different filtering techniques, and different envelope extraction techniques is also evaluated through a measurement setup consisting of a 20kg load cell and a multi-channel EMG sensor. This setup is subsequently used to estimate the force from the sEMG signals and compare it to the force measured using the load cell.
\chapter{Theory}
I will attempt to explain all theory required to understand this project using a top-down approach. By first explaining the big picture in which this research project is located we hope to avoid the reader getting lost in the details and misunderstanding the purpose of this report. 

\section{Force estimation}
When moving a limb the most intuitive way of describing it is a change in position, moving your hand from A to B. However, a more objective way of describing this movement is in terms of forces applied on a mass:
\begin{itemize}
    \item Muscles contract
    \item This causes a force to be applied on a mass, or a torque around a pivot point
    \item This force results in an acceleration in a certain direction
    \item This acceleration is maintained for a certain period of time
    \item The entire process is repeated for deceleration using visual feedback for fine tuning of forces
    \item Your limb has arrived at a new location.
\end{itemize}

Understanding this reasoning of moving a limb in terms of forces being applied by contracting muscles is vital because it forms the basis for recognizing a user's intent. EMG can be used to measure the degree of contraction of a muscle, and by measuring the degree of contraction of two antagonistic muscles it is possible to calculate how much force is applied in a certain direction which signals a desire for this limb to move. Even if the limb is not present and replaced by a prosthesis this idea of forces moving a mass will still apply, and thus EMG can be used as a human-machine interface.

So to summarize the basic concept of force estimation:
\begin{itemize}
    \item Movement of a limb is the result of a force acting on that limb
    \item This force is the result of certain muscles contracting stronger than other muscles
    \item The contraction of these muscles can be measured using EMG
    \item EMG can be used to estimate limb movement
\end{itemize}

\section{sEMG signal properties}

\begin{figure}[h!t]
	\begin{center}
		\includegraphics[height=60mm]{images/sEMG_signal_example.png}
	\end{center}
	\caption{sEMG signal measured from bicep during contraction}
	\label{fig:sEMG_signal_example}
\end{figure}

\begin{figure}[h!t]
	\begin{center}
		\includegraphics[height=60mm]{images/sEMG_fft_signalnoise_example.png}
	\end{center}
	\caption{Frequency components of signal and noise in an sEMG signal. 
    Noise is taken to be 0-2s and Signal is taken to be 5-7s in \ref{fig:sEMG_signal_example}}
	\label{fig:sEMG_fft_signalnoise_example}
\end{figure}

\begin{figure}[h!t]
	\begin{center}
		\includegraphics[height=60mm]{images/muscle_anatomy.png}
	\end{center}
	\caption{Anatomy of a muscle \cite{muscle_anatomy}}
	\label{fig:muscle_anatomy}
\end{figure}

\textcolor{red}{Todo: Find source that further explains muscles and EMG signals to refer to for further reading}

Figure \ref{fig:muscle_anatomy} illustrates the anatomy of a muscle which may be useful in this section. 
A large skeletal muscle such as the bicep consists of hundreds of thousands of small muscle fibers. These muscle fibers are divided into groups called motor units, and each motor unit is connected to a motor neuron which is a special type of very long brain cell that runs through the spinal cord. A contraction of a skeletal muscle is the result of many muscle fibers contracting individually and repeatedly. The contraction of these muscle fibers is the result of an action potential caused by the motor neuron, and this action potential is a measurable (but very small) voltage. When measuring the surface EMG of a contracting skeletal muscle the result is the aggregate of the small voltages from all contracting muscle fibers. This manifests itself into a signal resembling white noise where the amplitude of the noise correlates to the number of contracting muscle fibers and thus to the amount of contraction the skeletal muscle experiences. This is shown in \ref{fig:sEMG_signal_example} where a maximum voluntary contraction (MVC) is measured from a bicep.
 
So to summarize: to determine the degree of contraction of a skeletal muscle we simply need to determine the amplitude of the noise at the surface of the muscle.

From this point onwards I will refer to the 'noise' generated by muscle contraction as 'the signal'. This is done because noise is usually unwanted, but the signal caused by muscle contraction is the opposite of unwanted: It precisely what we're trying to measure! 

Unfortunately, when measuring sEMG signals it is impossible to measure solely the signal generated by muscle contraction. The signal may be polluted by other signals coming from the surrounding environment (such a 50 Hz power lines nearby) or from the amplifier used to amplify the measured signal. So in reality we are measuring a combination of our desired signal from muscle contraction, and the undesired noise from the environment and amplifier. An illustration of the frequency content of the signal and noise is shown in figure \ref{fig:sEMG_fft_signalnoise_example}. Note how the the noise has large peaks at 50Hz and multiples of 50Hz; This is the noise generated by the power lines. 

The ratio between how much of the measured signal is actual desired signal and how much is undesired noise is called the Signal to Noise Ratio (SNR) and is defined as the signal power divided by the noise power \textcolor{red}{(SOURCE) (EQUATION)}. As the amount of desired signal increases and the amount of noise decreases, the accuracy with which the force can be estimated from the contraction also increases. In other words, a larger SNR results in a more accurate estimation. We can increase the SNR by removing noise, and for this exact purpose 'filters' were invented.

\section{Filters}
% Introduce basic concept of filters
At a fundamental level filters are simply a tool to remove something unwanted (noise) that is mixed with something wanted (signal). In signal processing filtering is achieved by decomposing a measured signal into repeating patterns and subsequently deciding which patterns should be included and which patterns should be removed. Figure \ref{fig:filter_example} displays how a time-domain signal can be represented in the frequency domain to display informationa bout which frequencies are present in the signal.

\begin{figure}[h!t]
	\begin{center}
		\includegraphics[height=60mm]{images/filter_example.png}
	\end{center}
	\caption{Filtering a signal. In the top-left figure there is a low-frequency signal polluted by a 50Hz signal. The frequency plot in the bottom-left shows these frequencies. By applying the low-pass filter as displayed in the bottom-left it is possible to filter out the higher 50Hz frequency. The resulting filtered signal can be seen in the top-right, showing that there is still a little bit of noise left. This is also visible in the bottom right showing the frequency contents of the signal after filtering}
	\label{fig:filter_example}
\end{figure}

% Explain difference between analog digital and continuous and discrete
The physical world is inherently analog. Any property that changes from one value to another (such as a change in as wind speed, water level, voltage or age) will at some point have been equal to any value between the start and end value; Between 1 and 2 Volt there are infinite values with infinite decimals \textcolor{red}{(SOURCE)}. Computers sadly don't have infinite memory to store all these values and therefore to measure a continuous signal it needs to be sampled at equidistant points in time. This turns an analog continuous signal into a digital discrete signal. Even though filtering is possible in the analog domain, it is much less complicated and more flexible to filter in the digital domain \textcolor{red}{(NEED CITATION)}.

\textcolor{red}{< Image of analog vs digital as illustration of above text >}

In the digital domain the basic concept of a filter is as simple as it is magical. A filter (usually) consists simply of a set of values called the filter coefficients. The input (measurements) is multiplied with the filter coefficients to create the output. That is, the latest measurement is multiplied with the first filter coefficient, the previous measurement is multiplied with the second filter coefficient, and so on. This can also be interpreted as multiplying each filter coefficients with a delays input sample. Figure \ref{fig:wiki_digital_filter_working} shows the working and standard notation of a digital filter.
By carefully choosing the number and value of filter coefficients it is possible to attenuate specific frequencies while not influencing other frequencies such as the effect shown in \ref{fig:filter_example}.

\begin{figure}[h!t]
	\begin{center}
		\includegraphics[height=60mm]{images/wikipedia_fir_ditigal_filter.png}
	\end{center}
	\caption{The funcitoning of a digital filter. The filter coefficient at index $i$ is multiplied by the input that is delayed $i$ samples.}
	\label{fig:wiki_digital_filter_working}
\end{figure}

\subsection{Static filters, Wiener filter, Adaptive filters}
If a filter is static (e.g. high-pass, low-pass, band-pass, or band-stop) it simply means that the amount of filter coefficients and the values of the filter coefficients are predetermined. These filters are very popular due to their simplicity in terms of finding the value of the filter coefficients. If a signal is expected to be between 90 and 100 Hz, a bandpass filter can be created even before the signal is measured that removes all signal components \textit{except} those between 90 and 100 Hz just by multiplying some values together. 

\begin{figure}[h!t]
	\begin{center}
		\includegraphics[height=60mm]{images/Davis_intro_to_filters_filter_types.png}
	\end{center}
	\caption{Frequency responses of the common 4 static filters \cite{intro_to_static_filters}}
	\label{fig:static_filters}
\end{figure}

A Wiener filter is a static filter that removes the frequency contents of noise from the frequency contents of a signal. In other words, the Wiener filter attenuates the frequencies that are present in noise so that when the filter is applied to a signal that is polluted by noise with the same frequency contents it effectively removes the noise. This is done by measuring two things: The signal that is corrupted by noise, and the noise separately. For example when measuring an sEMG signal an additional channel is used to measure the signal of a non-contracting muscle to gather only the present noise. \textcolor{red}{(SOURCE)}
The Wiener filter requires both the signal and the noise to be stationary, i.e. the spectral density does not change over time, and results in a linear time-invariant filter \cite{wiki:Stationary_process} \cite{difference_stationary_nonstationary}.

\textcolor{brown}{\begin{itemize}
    \item Wiener-Hopf Equation
    \item Mean Square Error definition
    \item Least Square Error algorithm
\end{itemize}}

\textcolor{red}{< IMAGE WITH FREQUENCY RESPONSE OF WIENER FILTER > }

Adaptive filters expand on the idea of using knowledge of the spectral domain to create a filter by only using a subset of the input data (last $n$ samples). By re-calculating the filter coefficients with new data it is possible to update the filter to match a changing spectral density. 

Adaptive filters are applied in the same way as static filters by just multiplying the filter with measurements. However, the  method of finding the filter coefficients is different. Instead of deciding beforehand what frequencies you want to have removed, you measure the noise and determine its frequency spectrum. Then you tune the filter coefficients to remove these frequencies from the measurements (which also includes noise) leaving only the desired signal. This filter is used in environments where the noise frequency spectrum is not known beforehand, or when the frequency behaviour of the noise may change over time. Common applications of adaptive filters include speech recognition and noise-cancelling headphones! \cite{wiener_vs_adaptive_realtime_noisecancellation}

\textcolor{red}{(TODO: MATHEMATICAL DEFINITION OF FINDING COEFFICIENTS IN ADAPTIVE FILTERS) }

\textcolor{blue}{
=> LMS algorhitm
=> """"Instead of computing as suggested by
Wiener-Hopf equation, in LMS the coefficients are adjusted
from sample to sample in such a way as to minimize the MSE.... The LMS algorithm is based on steepest descent algorithm. ... """" => Wiener-hopf algorithm calculates $W_{opt}$ (Performance of Wiener Filter and Adaptive Filter for Noise Cancellation in Real-Time Environment, paper).}

\textcolor{red}{< IMAGE OF ADAPTIVE FILTER DIAGRAM >}

\subsection{FIR vs IIR}
Another subdivision within filter design is concerned with the type of possible responses to a specific input (impulse) and whether or not this can go to infinity.

The previously discussed filters were all described as Finite Impulse Response (FIR) filters. This means that the output is the result of multiplying the filter coefficients with the input. This type of filter is always stable and the output can never go to infinity as long as the input does not go to infinity.

An Infinite Impulse Response filter calculates the output using two sets of filter coefficients. The first set of filter coefficients is used to multiply with the input just like a FIR filter, but the second set of filter coefficients is used to multiply with previous \textit{outputs}. This means that there is now a feedback loop in the system, and a system with feedback can become unstable. Unstable in this case means that there is a possibility of positive feedback loop where increasing output values result in future output values also increasing, eventually going to infinity. Even though this feedback and possible instability may sound like a downside, it also results in shorter filter length and thus fewer computations required per filter operation. This could especially provide beneficial in low memory and low compute power environments like in prostheses \cite{fir_vs_iir}.

Both static and adaptive filters can be implemented as both FIR or IIR filters. An adaptive IIR filter offers the potential to meet desired performance levels with much less computational complexity. However, the possibility for the system to become unstable combined with the fact that filter coefficients are adjusted automatically leads to a high-risk high-reward scenario due to a loss of control and hard to predict behaviour. Background material on adaptive IIR materials can be found in chapter 23 of the Signal Processing Handbook \cite{digital_signal_processing_handbook}.

\begin{figure}[h!t]
	\begin{center}
		\includegraphics[height=60mm]{images/fir_vs_iir_diagram.png}
	\end{center}
	\caption{A diagram displaying the difference between Finite impulse response filters, only using previous input, and Infinite impulse response filters, using previous inputs and previous outputs resulting in a feedback loop \cite{fir_vs_iir_diagram}}
	\label{fig:fir_vs_iir_diagram}
\end{figure}


\section{Pre-whitening}
Back in 1948 a mathematician, electrical engineer, and cryptographer named C.E. Shannon published a pioneering paper that formed the basis of information theory \cite{shannon}. In this paper it is shown that repetition does not carry information, and that the maximum information transfer occurs when a signal is truly random. Imagine a signal with only a single frequency component. After measuring a few samples of the signal the conlusion is drawn that this is a 50Hz signal. Since it is possible to predict the value of every future measurement of the signal after drawing this conclusion it becomes unnecessary to continue measuring the signal because it will not give any new information. A repeating pattern is predictable, and predictable events carry no information.

The polar opposite of a signal containing a single frequency that is therefore predictable and carries little information is a signal that contains all frequencies an equal amount. This is called a white noise signal and carries the maximum amount of information because there exists no repetition and therefore every sample carries new, unpredictable information.

Between the existance of a signal containing a single frequency, and a signal containing all frequencys (white noise), all other signals exist and have certain frequencies that are more 'present' than other frequencies. These signals have different degrees of predictability (and thus information density), and the degree of predictability is determined by how closely the frequency content resembles white noise.

Whitening is a filtering technique that tries to equalize the presence of frequency components in a signal to approximate white noise and thus increase information density. It reduces the random error and yields a larger dynamic range because the small frequency components that contribute to the 'randomness' of the signal but not so much to the value of the measurement sample become more present \cite{time_series_analysis_methods} (Chapter: 5.4.9. Zero-Padding and Prewhitening). The serial correlation of the signal is decreased by reducing the presence of 'predictable' signals, which increases the randomness and thus information density \cite{serial_correlation_definition}. 

This previous information manifests itself in sEMG signal processing by the fact that the measured sEMG signal is not white. Some frequency components are much more present than others, but all frequencies equally contribute to the indication of muscle contraction. To get a more accurate indication of muscle contraction the signal should be whitened to increase the information of each sample.

Whitening in real-time is achieved through a digital filter with a frequency response that when multiplied with the sEMG signal frequency spectrum yields a white noise spectrum.

To summarize: Whitening reduces the power of repeating frequencies and increase the power of random frequencies in the signal. An example is given in figure \ref{fig:whitening_example}.

\begin{figure}[h!t]
	\begin{center}
		\includegraphics[height=60mm]{images/prewhitening_example.jpg}
	\end{center}
	\caption{An example of whitening a signal. The indicated peak contains the 'random' signal of interest. By whitening the powerful lower frequencies it is possible to give the information-carrying peak more presence on the signal \cite{time_series_analysis_methods}}
	\label{fig:whitening_example}
\end{figure}


\section{Envelope detection}

\begin{figure}[h!t]
	\begin{center}
		\includegraphics[height=60mm]{images/envelope_wikipedia.png}
	\end{center}
	\caption{Illustrating envelope detection of an analytical signal \cite{envelope_wikipedia}}
	\label{fig:envelope_wikipedia}
\end{figure}

\begin{figure}[h!t]
	\begin{center}
		\includegraphics[height=60mm]{images/amplitude_force_estimation_example.png}
	\end{center}
	\caption{On the left a time-domain sEMG signal. On the right an example of force estimation is presented. By taking the absolute value of the sEMG signal on the left, calculating the envelope, and recognizing that the force is a linearly scaled version of the envelope it is possible to estimate the force from the sEMG signal.}
	\label{fig:amplitude_estimation_example}
\end{figure}

As mentioned earlier the amplitude of a measured sEMG signal scales approximately linear with how much a skeletal muscle is contracted \cite{adaptive_filter_dry_electrode}. Since the raw EMG signal consists of stochastic and unpredictable noise it is difficult to draw conclusions about the degree of muscle contraction when solely looking at individual samples \cite{semg_signals_analysis_and_applications}. By drawing an outline of the peaks of the signal a much more informative picture can be drawn. This is called an envelope and an illustration of this process can be found in figure \ref{fig:envelope_wikipedia}. In the case of more random sEMG signals it is preferred to perform full wave rectification on the signal before calculating the envelope so that all of the signal energy is taken into account \cite{semg_signals_analysis_and_applications}. Applying this concept to an sEMG signal can be found in figure \ref{fig:amplitude_estimation_example}.

Computationally envelope detection can be achieved in a number of different methods where "speed", or how much the detected envelope lags behind the true signal envelope, is traded against accuracy or noisiness \cite{dsp_good_bad_ugly}. A few different envelope detection techniques are discussed in \cite{rose2011electromyogram}.

\subsection{Moving average}
A moving average filter is a special type of FIR filter with coefficients that are all have the value of $\frac{1}{n}$ where $n$ is the number of samples over which the average is taken. Thus the value of every smoothened sample is calculated to be the average of the previous $n$ samples. The upside of a moving average filter is that is is very simple to implement and introduces zero phase shift. A downside of a moving average filter is that it lags behind the signal by its very definition: a change in a static signal level is only properly reflected after $n$ samples. The sEMG signal must also be rectified before this method can be applied because EMG signal has approximately zero mean due to its oscillatory behaviour \cite{rose2011electromyogram}. 

\subsection{IIR Lowpass filter}
A lowpass filter such as a Butterworth or Chebyshev can be used to determine the envelope of a rectified signal in a more 'responsive' (less lag) method compared to a moving average filter. The downside of this filter is that it introduces phase shift unless applied in forward and backward direction \cite{rose2011electromyogram} which is not possible in real-time signals without introducing a static delay of a number of samples that equals the number of filter coefficients.

\subsection{Root Mean Square}
The Root Mean Square (RMS) of a signal is the square root of average power of a signal for a given period of time, a definition is given in equation \ref{eq:rms}. A useful property of RMS is that when it is applied to a signal with Gaussian distribution the RMS amplitude of the source is the same as the standard deviation of the distribution \cite{rms_standard_deviation}. In other words this means that RMS can extract the signal power of all frequencies in a signal in the time-domain if the frequencies are normally distributed.Since the probability density of surface EMG is approximately Gaussian, RMS should theoretically be the maximum likelihood estimator of EMG amplitude \cite{semg_signals_analysis_and_applications}.

\begin{equation}
    RMS = \sqrt{\frac{1}{n} (x[1]^2 + x[2]^2 + \cdots + x[n]^2)}
    \label{eq:rms}
\end{equation}

\section{Standard sEMG signal processing}\label{section:standard_semg_processing}
A conventional static real-time sEMG signal processing chain is described in \cite{muscle_force_estimation}. The relevant steps are as follows:
\begin{itemize}
    \item Highpass filter to remove DC
    \item Bandpass filter 20-300 hz
    \item Notch filter at 50hz
    \item Half-wave rectification
    \item Lowpass filter for envelope detection
\end{itemize}

This signal processing chain will also be tested in this report and compared to alternative techniques.
\chapter{Simulation}
This section of the report describes the testing of separate signal processing steps in a simulated environment. Each block as seen in figure \ref{fig:global_thesis_flowchart} will be tested individually, and the method and results will be discussion on a per-block basis:
\begin{itemize}
    \item Pre-whitening
    \item Filtering
    \item Envelope estimation
\end{itemize}

Unless specified otherwise all signals will be high-passed with a cutoff frequency of \SI{1}{\hertz} to remove DC bias before any operation is applied

\section{Pre-whitening}\label{sec:whitening}
A whitening filter is a digital filter with a frequency response that is (ideally) the inverse of the frequency contents of an sEMG signal. 

\subsection{Method}
A testing signal was created that approximates the frequency response of an sEMG signal. The testing signal was created by creating a long white noise signal and multiplying the amplitudes in the frequency spectrum with a curve that estimates the frequency response of an sEMG signal. After this the signal is passed through inverse fft to go back to a time-domain signal. The result can be seen in \ref{fig:whitening_simulation} subplot 1 and 2.

The whitening filter is then created by taking the FFT of the time-domain test signal and taking its reciprocal at every frequency. Lastly the whitening filter frequencies are multiplied by the mean of the original signal frequencies to make sure that when the filter is applied the mean stays the same. These steps can be seen in \ref{fig:whitening_simulation} subplot 3. 

\subsection{Results}
\begin{figure}[h!t]
	\begin{center}
		\includegraphics[width=1.0\columnwidth]{images/prewhitening_simulation.png}
	\end{center}
	\caption{Subplot 1 and 2 display the input simulated input signal with a frequency response that approximates the frequency contents of an sEMG signal. Subplot 3 displays the frequency content of the signal, the subsequently calculated whitening filter, and the multiplication of the signal with the filter in frequency domain to show that the response is indeed white. Subplot 4 and 5 show the original and 'whitened' signal in time domain and frequency domain.}
	\label{fig:whitening_simulation}
\end{figure}

In \ref{fig:whitening_simulation} it can be seen that the whitening filter functions as expected. In the center subplot the 'ideal' result can be seen (multiplication in the frequency domain), and in subplot 5 the result from convolution in the time domain is shown. In subplot 5 a smoothened version of the filtered signal FFT is added to display that the signal power has a mean that approximates white noise. This filter was made using a Savitzky-Golay filter with a window length of 31 and a polynomial order of 3.

\section{Filtering}
Due to the difficulty of creating a simulated signal that has the frequency contents of an sEMG signal and environmental noise, a pre-recorded sample of sEMG of a bicep going through maximum voluntary contraction was used to test the filters. This sample is not used for force estimation because there is no way to validate the degree of contraction, and purely the frequency contents of the signal are of interest. The sample that is used can be seen in figure \ref{fig:sEMG_signal_example}. 

The construction of each filter will be explained in the method, and the SNR performance and bandwidth of each filter will be discussed in the results. 

\subsubsection{Static filter}
The theory from section \ref{section:standard_semg_processing} specifies the removal of DC frequencies, a notch filter at \SI{50}{Hz}, and a bandpass filter between \SI{20}{Hz} and \SI{300}{Hz}. Looking at the noise spectrum in figure \ref{fig:sEMG_fft_signalnoise_example} it can be seen that there also exist significant peaks at \SI{100}{Hz} and \SI{150}{Hz}. For this reason, additional notch filters are constructed to target these frequencies. The total frequency response of the concatenated filters can be seen in figure \ref{fig:staticfilter_frequencyresponse}.
\begin{itemize}
    \item IIR Notch filters at \SI{50}{Hz}, \SI{100}{Hz}, \SI{150}{Hz}. All have a Q-factor of 10, are constructed as numerator/denominator pairs and applied using scipy's lfilter.
    \item The bandpass filter consists of a highpass filter with a cut-off frequency of \SI{20}{Hz} and a lowpass filter with a cut-off frequency of \SI{300}{Hz}. Both filters are of length 5, are constructed as numerator/denominator pairs, and applied using scipy's lfilter.
\end{itemize}

\subsection{Method}
\begin{figure}[h!t]
	\begin{center}
		\includegraphics[width=1.0\columnwidth]{images/staticfilter_frequencyresponse.png}
	\end{center}
	\caption{Frequency response of the static filter. This plot was created by determining the frequency responses of each individual filter, then multiplying the amplitudes and adding the phase shifts}
	\label{fig:staticfilter_frequencyresponse}
\end{figure}


\subsubsection{Wiener filter}
As discussed in section \ref{sec:filters_theory} the Wiener filter coefficients are constructed from the crosscorrelation vector between signal+noise ($d(n)$) and the noise ($v(n)$) and the autocorrelation of the noise as is presented in the Wiener-Hopf equation in equation \ref{eq:wiener_hopf} \cite{lecture_adaptive_filters_1}. 

The number of Wiener filter coefficients have a strong influence on the performance of the filter as can be seen in figure \ref{fig:wiener_filter_length}. The explanation behind SNR calculation and bandwidth will be discussion in the 'Results' section.

\begin{equation}
    w_{opt} = R^{-1}P
    \label{eq:wiener_hopf}
\end{equation}

\begin{figure}[h!t]
	\begin{center}
		\includegraphics[width=1.0\columnwidth]{images/wiener_filter_length.png}
	\end{center}
	\caption{The effect of the number of Wiener filter coefficients on the SNR and bandwidth}
	\label{fig:wiener_filter_length}
\end{figure}

\subsubsection{Adaptive Wiener filter}
The method of calculation to obtain the filter coefficients for the adaptive Wiener filter is identical to the one presented in the Wiener filter. The difference with this method is that instead of taking a large signal sample and noise sample to calculating the filter coefficients and using those on the entire signal, the filter coefficients are repeatedly recalculated over the latest sample set. This allows the filter to adapt to changing statistical signal properties. This leaves the question: How often should the coefficients be recalculated and over how many samples should the filter coefficients be calculated?

Using the previously discussed prerecorded sample that can be seen in figure \ref{fig:sEMG_signal_example} various combinations of window size (number of samples from which the coefficients are calculated) and 'blocks' per window (how many samples are skipped before recalculating the filter coefficients) were tested. The results can be found in figure \ref{fig:adaptive_wiener_windowsize}. For the comparison between different filters, a window length of 500 was used in combination with 1 block per window. In other words, every 500 samples the Wiener filter coefficients are calculated using the last 500 samples.

\begin{figure}[h!t]
	\begin{center}
		\includegraphics[width=1.0\columnwidth]{images/adaptive_wiener_windowsize.png}
	\end{center}
	\caption{SNR and bandwidth of an adaptive wiener filter using different combinations of window size and blocks per window. It should be noted that each signal has been smoothened using a Savitzky-Golay filter with a window length of 31 and a polynomial order of 3. This is done to make the difference in performance between different blocks per window clearer, without filtering the signals have a larger deviation that makes the lines unreadable.
	It can be seen that the number of blocks per window does not have a big influence on the SNR or bandwidth. The SNR seems to have a somewhat linear relation with the window size, while the bandwidth peaks around a window size of 500 samples.}
	\label{fig:adaptive_wiener_windowsize}
\end{figure}


\subsection{Results}
Each filter will be tested using two metrics: SNR (eq \ref{eq:rms}) and Bandwidth (definition of this will be given shortly). All filters are linear time invariant which means the superposition principle can be used to simplify SNR calculations \cite{linear_systems_theory}. The superposition principle simply states that filtering the sum of two signals is the same as filtering the signals individually and adding the results. An illustration of this can be seen in \ref{fig:filter_process}. A sample of noise data and a sample of EMG data is taken from \ref{fig:sEMG_signal_example} where noise is taken to be 0-2s and signal is taken to be 5-\SI{7}{\second}. The RMS of the filtered signal is divided by the RMS of the filtered noise to create a signal to noise ratio.

\begin{figure}[h!t]
	\begin{center}
		\includegraphics[width=1.0\columnwidth]{images/filter_process.png}
	\end{center}
	\caption{Illustration of testing of filters. Signal and noise are passed through the filter invidiually and the SNR is calculated for each filter.}
	\label{fig:filter_process}
\end{figure}

A property that might be of interest is each filters performance in different levels of Maximum Voluntary Contraction (MVC). This allows for insight into how well each filter functions in different levels of signal compared to the environment noise. This was realized by keeping the noise constant, but scaling the signal to different levels (from 1-\SI{100}{\percent}) to simulate different levels of MVC. The SNR of the filtered signal and filtered noise was divided by the reference SNR (SNR of input signal and input noise) to be able to draw a clear conclusion about the filters performance.

\begin{figure}[h!t]
	\begin{center}
		\includegraphics[width=0.7\columnwidth]{images/filter_snr_mvc.png}
	\end{center}
	\caption{The SNR of each filter for different levels of MVC. It can be seen that there is no relation between a filters performance and the degree of contraction.}
	\label{fig:filter_snr_mvc}
\end{figure}

SNR by itself is not a valid metric for judging a filters performance in this scenario. The purpose of improving SNR is the assumption that force can be estimated more accurately from a signal that contains primarily the signal generated by muscle contraction. However, a filter may be able to attenuate the signal and noise in such a way that the SNR is very high, but the signal is attenuated to such a degree that it no longer resembles the original signal that was generated by the muscle contraction. An example of this can be seen in figure \ref{fig:good_snr_bad_integrity}. A measure to define how much the frequency spectrum has changed is the bandwidth. Typically the bandwidth of a signal is defined as the range of frequencies between two frequency points outside of which the signal is attenuated more than a specific threshold value \cite{bandwidth_definition}. This definition is not applicable to this problem as the frequencies that are 'present' in an sEMG signal are not necessarily consecutive. Therefore the bandwidth of an sEMG signal will be defined as the number of frequency components that are larger than the mean of the frequency spectrum. With this metric, a frequency spectrum such as the one seen in figure \ref{fig:good_snr_bad_integrity} will have a low bandwidth because most frequencies are below the mean.

\begin{figure}[h!t]
	\begin{center}
		\includegraphics[width=1.0\columnwidth]{images/good_snr_bad_integrity.png}
	\end{center}
	\caption{Example of filtering that results in good SNR but bad bandwidth.}
	\label{fig:good_snr_bad_integrity}
\end{figure}

Again, the bandwidth was calculated for different filters and at different levels of MVC. The results can be seen in figure \ref{fig:filter_bw_mvc}.

\begin{figure}[h!t]
	\begin{center}
		\includegraphics[width=0.7\columnwidth]{images/filter_bw_mvc.png}
	\end{center}
	\caption{The bandwidth of each filter for different levels of MVC. It can be seen that there is no relation between a filters performance and the degree of contraction}
	\label{fig:filter_bw_mvc}
\end{figure}

For the static filter a few different variations were tested. Looking at figure \ref{fig:sEMG_fft_signalnoise_example} it can be seen that beside the \SI{50}{\hertz} peak in the noise spectrum there are additional peaks at multiples of \SI{50}{\hertz}. Different static filters were tested with different amounts of notch filters. The frequency response of these filters can be seen in figure \ref{fig:staticfilter_notches_frequencyresponse}. The resulting metrics can be seen in the chart  \ref{fig:staticfilter_notches_barchart}.

\begin{figure}[h!t]
	\begin{center}
		\includegraphics[width=1.0\columnwidth]{images/staticfilter_notches_frequencyresponse.png}
	\end{center}
	\caption{Frequency response of static filters with different amounts of notch filters. For the sake of illustration the amplitude graphs have been shifted vertically to clearly show the existence of notch filters in different lines, during simulations this shift was not present. }
	\label{fig:staticfilter_notches_frequencyresponse}
\end{figure}

\begin{figure}[h!t]
	\begin{center}
		\includegraphics[width=1.0\columnwidth]{images/staticfilter_notches_barchart.png}
	\end{center}
	\caption{SNR and bandwidth of static filters with different numbers of Notch filters. }
	\label{fig:staticfilter_notches_barchart}
\end{figure}

\section{Envelope estimation}
The construction of different envelope estimation techniques will be discussed in the method section. The metric for testing performance with be discussed in the results section, as well as the results themselves.

\subsection{Method}
\subsubsection{IIR lowpass filter}
A Butterworth filter was used to construct an IIR lowpass filter. The performance of such a lowpass filter is defined by its cut-off frequency and the number of filter coefficients (or the filter order). It was empirically determined that the maximum frequency for switching between total relaxation and maximum voluntary contraction was around \SI{5}{\hertz} and thus the cut-off frequency was varied from \SI{1}{\hertz}-\SI{9}{\hertz}. The filter order was varied from 2-8 because the minimum possible filter order is 2 (a single filter coefficient provides no filtering, just scaling the signal with a constant), and a filter order >8 resulted in unstable behaviour. A plot of the frequency response of the IIR Butterworth filter can be seen in figure \ref{fig:iir_frequencyresponse_coefficients}. The filters are achieved as a numerator/denominator sequence. Since the purpose of this filter is real-time envelope detection, it was applied using scipy's lfilter since as that is causal forward-in-time filtering only.


\begin{figure}[h!t]
	\begin{center}
		\includegraphics[width=1.0\columnwidth]{images/iir_frequencyresponse_coefficients.png}
	\end{center}
	\caption{Frequency response of IIR Butterworth filter of different lengths. The cut-off frequency was set to 5Hz.}
	\label{fig:iir_frequencyresponse_coefficients}
\end{figure}

\subsubsection{Moving average}
The moving average filter only depends on the length of the filter, figure \ref{fig:movingaverage_frequencyresponse_coefficients} depicts the frequency behaviour of the moving average filter of different lengths. The range of values that are tested is chosen arbitrarily, but large enough to cover general use cases.

\begin{figure}[h!t]
	\begin{center}
		\includegraphics[width=1.0\columnwidth]{images/movingaverage_frequencyresponse_coefficients.png}
	\end{center}
	\caption{Frequency response of moving average filter of different lengths. The coefficients of the moving average filters are 1/length of filter.}
	\label{fig:movingaverage_frequencyresponse_coefficients}
\end{figure}

\subsubsection{Root mean square}
Similar to the moving average filter, the behaviour of the RMS filter is solely determined by the length of the filter. The same range of filter lengths was chosen as for the moving average filter so that the performance could be directly plotted against each other.

\subsection{Results}
There are two seperate metrics that need to be measured when comparing envelope estimation techniques. The first one is how 'fast' a techniques is, and the second is how 'good' the technique is. The first metric gives information about how much the 'detected' signal lags behind the 'true' signal. The second metric is the quality of the envelope estimate when accounting for the lag.

The lag is detected by calculating the cross-correlation between the true signal and the envelope estimate. Cross-correlation is a metric that determines the similarity between two signals as a function of displacement of one signal relative to another \cite{wiki:cross_correlation}. Since the 'true' signal and the estimated signal are most similar when their displacement equals the lag, a detectable peak is formed in the cross correlation function. The left subplot in figure \ref{fig:envelope_estimation_method} displays a 'true' signal (measured force), and a simulated estimation of this signal (estimated force) that lags behind the true signal. The cross-correlation is also plotted that has a peak at \SI{100}{\milli\second}, which is the exact amount of lag between the two signals. The right subplot shows the two signals where the estimated signal has shifted to account for the lag. At this point the error can be determined by subtracting the true signal from the estimated signal, and the root-mean-square-error can be calculated. 

\begin{figure}[h!t]
	\begin{center}
		\includegraphics[width=1.0\columnwidth]{images/envelope_estimation_method.png}
	\end{center}
	\caption{Illustration of method for judging envelope estimation}
	\label{fig:envelope_estimation_method}
\end{figure}

The input signal was generated to be gaussian white noise since it has signal properties close to that of an sEMG signal, and is multiplied with a modulation that can be seen in the left subplot of figure \ref{fig:envelope_detection}. The envelope detection techniques are applied onto the input signal and the results are plotted in the right subplot of figure \ref{fig:envelope_detection}.

\begin{figure}[h!t]
	\begin{center}
		\includegraphics[width=1.0\columnwidth]{images/envelope_detection.png}
	\end{center}
	\caption{Left: Input signal and 'true' envelope'. Right: Envelope detection using different techniques to illustrate difference in behaviour}
	\label{fig:envelope_detection}
\end{figure}

To properly evaluate the performance of the envelope detection techniques each method has been tested individually across the range of variables that were described in the previous method section and plotted against the resulting lag and error. The graph describing the performance of the IIR butterworth filter can be seen in figure \ref{fig:lagerror_iir}. The graph describing the performance of the moving average filter and the RMS filter can be seen in figure \ref{fig:lagerror_RMS_MA}.

\begin{figure}[h!t]
	\begin{center}
		\includegraphics[width=1.0\columnwidth]{images/lagerror_iirfilter.png}
	\end{center}
	\caption{Lag and error of an IIR butterworth filter for different cut-off frequencies and filter lengths. Note that filters with a lower cut-off frequency and high number of filter coefficients become unstable which can be seen in the error-graph for cut-off frequencies \SI{1}{\hertz}-\SI{3}{\hertz}. Additionally, the error in these graphs are all below 0.125. This is caused by the fact that the modulation is between 0 and 1, the error is <1 and the squared error is smaller still. So not the error value, but the \textit{relation between} = error values of different methods is the truly useful information here.}
	\label{fig:lagerror_iir}
\end{figure}

\begin{figure}[h!t]
	\begin{center}
		\includegraphics[width=1.0\columnwidth]{images/lagerror_rms_and_MA_filter.png}
	\end{center}
	\caption{Lag and error of RMS filter and moving average filter for different filter lengths}
	\label{fig:lagerror_RMS_MA}
\end{figure}

\section{Force estimation}\label{section:force_estimation}

The measured sEMG signals from antagonistic muscles needs to be combined to form an estimate of the exerted force. Since this calculation is done in a consistent way throughout all measurements this section serves purely to provide some insight into the method of calculation.

The previous step of envelope estimation is used to get a measure of muscle contraction. Exerted force around a joint is simply the difference between how much antagonistic contract. By measuring the sEMG from antagonistic muscles (such as bicep and tricep) and determining their envelope after processing, the estimated force is simply the difference between the two envelopes. However, this method does not directly result in an estimation of force but actually an estimation in the difference of muscle contraction. To correlate the difference in muscle contraction from antagonistic muscles to the exerted force, the muscle contraction needs to be scaled. As previously mentioned, a linear relation between sEMG and force will be assumed \cite{adaptive_filter_dry_electrode} \cite{interpreting_muscle_function_from_emg}. In figure \ref{fig:force_simulation} it is shown how the exerted force is estimated from simulated bicep and tricep sEMG. 

\begin{figure}[h!t]
	\begin{center}
		\includegraphics[width=1.0\columnwidth]{images/force_simulation.png}
	\end{center}
	\caption{Process of estimating force from simulated sEMG (random Gaussian noise). The bicep envelope is calculated at the top, notice how during downwards force exertion the bicep still activates but to a lesser degree than during upwards force exertion. The same holds for the tricep in the bottom figure. The subplot on the right shows the envelope of the bicep, tricep, the difference between these two (identical to estimated force assuming linear scaling with factor 1), and the 'measured' force .}
	\label{fig:force_simulation}
\end{figure}


\chapter{Measurements}
This chapter aims to validate the accuracy of force estimation from sEMG after different processing techniques by comparing it to measured data. The goal is to measure sEMG from biceps and triceps, and record sEMG reference noise, and measure the estimated force using a load cell during an exercise of isometric contraction. The sEMG data is then processed using the different techniques that are discussed in the simulation chapter, and the final estimated force will be compared to the measured force. 

\section{Experimental setup}
The measurement setup consists of the following components:
\begin{itemize}
    \item Siemens Single Point Load Cell, 20kg Range, Compression Measure
    \item Keysight E3631A DC power supply at 2V to power the load cell
    \item TMSi Refa8-16e 16 channel amplifier
    \item Kendall H124SG Foam-Hydrogel ECG Electrodes 
\end{itemize}




Load cell with handle
Power supply
Multi-channel amplifier




The actual measuring of force data. Mention that 600g was used in an initial experiment but it yielded odd data (list possible sources: muscle contraction was very little, lot of noise in room, shield maybe not working, LOFF not working, too many channels?), so a heavier load cell was used for more correct estimation.

What did I measure, how did I measure it, where did I measure it. What did I record, what do I expect to see?

Mention hardware used (adc, load cell,) and calibration tests done.
\section{Measurement results}
Data that was NOT working, data that WAS working, notable characteristics of the data


\section{Conclusion}
Draw numeric conclusion from data, in which situations is which filter how much better?
\chapter{Discussion and conclusion}
\section{Discussion}
The research question presented in the introduction states the goal of finding the best combination of whitening, filtering, and envelope detection to estimate force from sEMG. 

From the measurement results presented in figure \ref{fig:result_all_lagerrorscaling} the conclusion can be drawn that a processing chain consisting of no pre-whitening, no filtering, and infinite impulse response butterworth lowpass filter for envelope detection yields the best accuracy for force estimating. It has the lowest error rate of all techniques and introduces no lag.

This indicates that improving the signal to noise ratio of an sEMG signal does not result in a better estimation of the applied force. It should be noted that these conclusions are drawn from a single measurement at a single location and thus may yield different results in different environments. 

Another interesting observation can be made from the Lag comparison subplot in figure \ref{fig:result_all_lagerrorscaling}: only signal processing chains that use whitening appear to introduce lag. This could be caused by the phase delay that is introduced by the whitening filter. If the filter were to have a linear phase then all frequencies would be equally delayed resulting in a constant group delay. If this filter did not have linear phase but instead introduces a stronger phase delay in higher frequencies, then the total delay would become more apparent when amplifying these higher frequencies \cite{phase_delay_frequencies}. The amplification of higher frequencies does happen in the whitening filter as can be seen in the simulation in \ref{fig:whitening_simulation}, and the time domain plot shows signs of introduction of delay in the filtered signal which can stem from making higher frequencies with more phase delay more prominent. In the simulations the phase of the whitening filter was set to the phase of the input signal ('source' of filter). In the appendix are the results of constructing the whitening filter using zero-phase, linear-phase, and negative input signal phase \ref{fig:result_whitening_linearphase} \ref{fig:result_whitening_negative_sourcephase}\ref{fig:result_whitening_sourcephase}\ref{fig:result_whitening_zerophase}. Intuitively it would be expected that using negative source phase would counteract this introduced delay, but as can be seen in appendix figure \ref{fig:result_whitening_negative_sourcephase} this does not appear to be the case.

A reason why other processing steps do not seem to introduce delay when not paired with whitening could be that lag is calculated using the cross-correlation between the measured force and the estimated force. If the estimated force accurately follows the measured force, there would exist a clear peak in the cross-correlation which indicates the lag.




=> Other signals noise is larger than lag 


=> Combination of adaptive + whitening == very very bad
- instability of IIR filter

=> Beantwoorden van onderzoeksvraag

=> We make this conclusion but that only holds under these cirumstances

describe what your results mean and how they are an important contribution to the research field.

- short signal sample of MVC and noise
- no load cell measurement


What does it all mean?
◼ What hypotheses were proved or disproved?
◼ What did we learn?
◼ Why is it important (enough to report)?
\section{Conclusion}
In this report a complete signal processing chain for sEMG signals was presented. Within each step in this processing chain various processing techniques were discussed and tested to illustrate their behaviour and performance when applied to sEMG signals. 



Summarize your key findings. Include important conclusions that can be drawn and further implications for the field. Discuss benefits or shortcomings of your work and suggest future areas for research.

High-level conclusion that was not previously mentioned (The adaptive filter is better in X situation but at the cost of Y)

Summarize general trends in the data without comment, bias, or interpretation.
◼ Should add a new, higher level of analysis, and explicitly indicate the importance
of the work
◼ Do not repeat the Results section, unless special emphasis is needed
◼ Conclusions are not a summary of the work!

=> Samenvatting discussie met alleen de belangrijkste punten

\section{Recommendations}
Further research, suggestions for better results


% Appendix starts here
% change file name for better descriptive names, but start with apx-
\appendix
\chapter{Appendix 1} \label{app:whitening}

\begin{figure}[h!t]
	\begin{center}
		\includegraphics[width=1.0\columnwidth]{images/result_whitening_sourcephase.png}
	\end{center}
	\caption{Lag, error, and scaling of different filtering and envelope techniques with whitening applied. The whitening filter is constructed from the desired frequency amplitude response and a phase. The frequency response is determined as described in the simulation section \ref{sec:whitening}, and the phase is set to the phase of the input signal that was used to construct the whitening filter.}
	\label{fig:result_whitening_sourcephase}
\end{figure}

\begin{figure}[h!t]
	\begin{center}
		\includegraphics[width=1.0\columnwidth]{images/result_whitening_linearphase.png}
	\end{center}
	\caption{Lag, error, and scaling of different filtering and envelope techniques with whitening applied. The whitening filter is constructed from the desired frequency amplitude response and a phase. The frequency response is determined as described in the simulation section \ref{sec:whitening}, and the phase is set to linear phase.}
	\label{fig:result_whitening_sourcephase}
\end{figure}

\begin{figure}[h!t]
	\begin{center}
		\includegraphics[width=1.0\columnwidth]{images/result_whitening_zerophase.png}
	\end{center}
	\caption{Lag, error, and scaling of different filtering and envelope techniques with whitening applied. The whitening filter is constructed from the desired frequency amplitude response and a phase. The frequency response is determined as described in the simulation section \ref{sec:whitening}, and the phase is set to zero phase.}
	\label{fig:result_whitening_sourcephase}
\end{figure}

% Bibliography starts here
\backmatter

% Generate bibliography
\fancyhead[LO]{Bibliography}
%\bibliographystyle{include/files/RaM-bibtex}
\bibliographystyle{ieeetr}
\label{ch:bib} %label to refer to
\bibliography{bibliography} 

\end{document}

